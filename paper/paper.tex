\documentclass[conference]{IEEEtran}

\usepackage[fleqn]{amsmath}
\usepackage{algorithmic}
%\setlength{\mathindent}{0pt}
\interdisplaylinepenalty=2500
\hyphenation{op-tical net-works semi-conduc-tor}

\begin{document}
\title{Bat Optimization on GPU}

\author{\IEEEauthorblockN{Jean Carlo Machado}
\IEEEauthorblockA{Univerisdade do Estado de Santa Catarina\\Mestrado de Computação Aplicada\\
UDESC\\
Santa Catarina, Joinville,\\
Email: contato@jeancarlomachado.com.br}}

\maketitle
\begin{abstract}
    This works presents the Bat in GPU.
    The results show that...
\end{abstract}
\IEEEpeerreviewmaketitle

\section{Introduction}

The bat algorithm was introduced by Yang \cite{original}. It uses the
inspiration of micro-bats to look at their preys.

Many populational optimization algorithms can benefit from
parallelization.

This work attempts to investigate the applicability of GPU parable
libation on the bat algorithm. Previously some demonstrations of the
bat algorithm parallelized on CPU were presented (reference reference),
however, til the day of this publication no implementation of the bat
algorithm was found on GPU.

It was developed two versions of the algorithm. One that runs on CPU and
the other which uses CPU.

\hfill December 21, 2016

In this work the bath algorithm used was the one proposed by
\cite{parpinelli}, since it represents a concrete demonstration of how
the bat metaheuristic given that the original papers lacks it.

\section{Bat Design on CPU}

The CPU version developed was single threaded.
The random algorithm used was the mersenne twister.

\subsection{Pseudo-code}

\begin{flalign}
\begin{split}
\label{Pseudo Code}
\text{Parameters:} n, \alpha\, \lambda\\
\text{Initialize Bats}\\
\text{Evaluate fitness bats}\\
\text{Selects best}\\
\text{\textbf{while} stop criteria false \textbf{do}}\\
    \text{\textbf{for} i=1\ to\ n\ \bf do}\\
        f_i=f_{min} + (f_{max} - f_{min})\beta, \in \beta [0,1]\\
        \vec{v}_i^{t+1} = \vec{v}_i^{t} + (\vec{x}_i^{t} + \vec{x}_*^{t})f_i\\
        \vec{x}_{temp} = \vec{x}_i^{t} + \vec{v}_i^{t+1}\\
        \text{\bf if } rand < r_i, rand \in [0,1] \text{\textbf{then} {local search}}\\
            \vec{x}_{temp} = \vec{x}_* + \epsilon A_m, \epsilon \in [-1, 1]\\
        \text{\bf end if}\\
        \text{Single dimension perturbation in} x_{temp}\\
        \text{\bf if} a < A_i^t , a \in [0,1]\\
        \vec{x}_i^t = \vec{x}_{temp}\\
        r_i = exp(\lambda * i)\\
        A_i =  A_{0} * \alpha^i\\
        \text{\bf end if}\\
        \text{Selects best}\\
    \text{\bf end for}\\
\text{\bf end while}
\end{split}
\end{flalign}

\section{Bat Design on GPU}

Since the BAT algorithm uses a population of bats, the most intuitive
parallelization method to apply on it is to use each bat on a GPU core.
\cite{pso-gpu} used a similar method for a GPU implementation for the PSO algorithm.
For the GPU version the approach used was the split of each individual in one thread.

\subsection{Pseudo-code GPU}
\begin{algorithmic}
\label{Pseudo Code}
\STATE Parameters: n, \alpha\, \lambda
\STATE Initialize bats assycrously
\STATE synchronize threads
\STATE Evaluate fitness bats
\STATE Selects best
\STATE \textbf{while stop criteria false \textbf{do}}
    \STATE for each thread  i 
        f_i=f_{min} + (f_{max} - f_{min})\beta, \in \beta [0,1]\\
        \vec{v}_i^{t+1} = \vec{v}_i^{t} + (\vec{x}_i^{t} + \vec{x}_*^{t})f_i\\
        \vec{x}_{temp} = \vec{x}_i^{t} + \vec{v}_i^{t+1}\\
        \STATE \bf if  rand < r_i, rand \in [0,1] \textbf{then} {local search}\\
            \vec{x}_{temp} = \vec{x}_* + \epsilon A_m, \epsilon \in [-1, 1]\\
        \STATE \bf end if
        \STATE Single dimension perturbation in x_{temp}
        \STATE \bf if a < A_i^t , a \in [0,1]
        \vec{x}_i^t = \vec{x}_{temp}\\
        r_i = exp(\lambda * i)\\
        A_i =  A_{0} * \alpha^i\\
        \STATE \bf end if
        \STATE synchronize threads
        \STATE Selects best
    \STATE \bf end for
\STATE \bf end while
\end{algorithmic}

\section{Experiments}

The benchmark functions used were the following:

\begin{itemize}
    \item Ackley
    \item Griewank
    \item Rastringin
    \item Rosenbrook
\end{itemize}

The experiments were executed on a machine with the following configuration:

\textit{Intel(R) Core(TM) i5-4460  CPU @ 3.20GHz \\ GK208 GeForce GT 720 1024 MB of vram}

Each experiment runned a total of 20 times with 10 thousand iterations each.

\begin{table}[!t]
    \renewcommand{\arraystretch}{1.3}
    \caption{Experiments}
    \label{experiments}
    \centering
    \begin{tabular}{c|c|c|c}
    \hline
        \bf Name & Function &  Dimensions & Agents\\
    \hline
        E1 & Ackley & 100 & 256\\
        E2 & Ackley & 100 & 768\\
        E3 & Griewank & 100 & 256\\
        E4 & Griewank & 100 & 768\\
        E5 & Rastringin & 100 & 256\\
        E6 & Rastringin & 100 & 768\\
        E7 & Rosenbrook & 100 & 256\\
        E8 & Rosenbrook & 100 & 768\\
    \end{tabular}
\end{table}

\section{Results}

\begin{table}[!t]
    \renewcommand{\arraystretch}{1.3}
    \caption{Results}
    \label{results}
    \centering
    \begin{tabular}{c|c|c|c|c}
    \hline
        \bf Name & Fitness C & Total C & Fitness G & Total & Speedup\\
    \hline
        E1 & 57.3774s & 1.69691e-06 &\\
    \end{tabular}
\end{table}

\section{Conclusion}

It was observed speedups with big populations. The original BAT was
proposed with 40 individuals and the speedups was seen with 250
individuals. The advantages of the algorithm may be tested against a threaded CPU
implementation to be fair. With this work it's clear that is possible to
speedup the bat metaheuristic using GPU but the best results are only
achievable on really complex problems with many dimensions.

\section{Further works}

It may be explored the usage of blocks as representation for the
dimensions in which each bat details.

\begin{thebibliography}{1}

\bibitem{original}
    Xin-She Yang \emph{A New Metaheuristics Bat-Inspired Algorithm}. Department of Engineering, Cambridge, 2010.
\bibitem{parpinelli}
    Jelson A. Cordeiro, Rafael Stubs Parpinelli Heitor Silvério Lopes \emph{Análise de Sensibilidade dos Parâmetros do Bat Algorithm e Comparação de Desempenho}.
\bibitem{pso-gpu}
    PSO-GPU: Accelerating Particle Swarm Optimization in CUDA-Based Graphics Processing Unit
\end{thebibliography}

\end{document}
