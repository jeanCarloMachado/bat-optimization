\documentclass[12pt]{article}

\usepackage{sbc-template}

\usepackage{graphicx,url}

%\usepackage[brazil]{babel}   
\usepackage[latin1]{inputenc}  


\sloppy

\title{Bat Algorithm on GPU}

\author{Jean Carlo Machado\inst{1}}

\address{ Universidade Estadual de Santa Catarina
  (UDESC) Joinville, SC.
}

\begin{document}

\maketitle

\begin{abstract}
\end{abstract}

\begin{resumo} 
\end{resumo}


\section{General Information}

The bat algorithm was introduced...

In this work the bath algorithm used was the one proposed by (parpinelli reference).

It was developed two versions of the algorithm. One that runs on CPU and the other which uses CPU.



The benchmark functions used were the following:

ROSENBROOK, SPHERE, SCHWEFEL, ACKLEY, RASTRINGIN, GRIEWANK, SHUBER

For the GPU version the approach used was the split of each individual in one thread.
It was observed speedups with big populations.


\section{Experiment}

Experiment 1

Function: Griewank
Numero de indivíduos: 256
Numero de iteracoes: 10000


Experiment 2

Function: Griewank
Numero de indivíduos: 768
Numero de iteracoes: 10000



\section{Results}



\subsection{Device details}

The experiments were executed on a machine with the following configuration:

Intel(R) Core(TM) i5-4460  CPU @ 3.20GHz

GK208 GeForce GT 720

\subsection{E1}

1m4.888s
0m55.439s
Speedup: 1.16x


\subsection{E2}

2m27.902s
0m21.976s
Speedup: 7x


\section{References}

Bibliographic references must be unambiguous and uniform.  We recommend giving
the author names references in brackets, e.g. \cite{knuth:84},
\cite{boulic:91}, and \cite{smith:99}.


\bibliographystyle{sbc}
\bibliography{sbc-template}

\end{document}
