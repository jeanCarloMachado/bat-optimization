\documentclass[12pt]{article}

\usepackage{sbc}

\usepackage{graphicx,url}

%\usepackage[brazil]{babel}   
\usepackage[latin1]{inputenc}  


\sloppy

\title{Bat Algorithm on GPU}

\author{Jean Carlo Machado\inst{1}}

\address{ Universidade Estadual de Santa Catarina
  (UDESC) Joinville, SC.
}

\begin{document}

\maketitle

\begin{abstract}
\end{abstract}

\begin{resumo}
\end{resumo}


\section{General Information}

The bat algorithm was introduced by \cite{original}. It uses the inspiration of microbas to look at their preys.

Many populational optimization algorithms can benefit from paralization.

This work attemps to investigate the applicability of GPU parallization
on the bat algorithm. Previously some demonstrations of the bat
algorithm paralelized on CPU were presented (reference referece),
however, til the day of this publication no implementation of the bat
algorithm was found on GPU.

In this work the bath algorithm used was the one proposed by
\cite{parpinelli}, since it represents a concrete demonstration of how
the bat metha-heuristic.

It was developed two versions of the algorithm. One that runs on CPU and
the other which uses CPU.

\section{Bat Design on CPU}

The CPU version was single threaded.

\section{Bat Design on GPU}

Since the BAT agorithm uses a population of bats, the most intuitive
parallization method to apply on it is to use each bat on a GPU core.
\cite{pso-gpu} used a similar method for a GPU implementation for the PSO algorithm.

For the GPU version the approach used was the split of each individual in one thread.

The benchmark functions used were the following:

ROSENBROOK, SPHERE, SCHWEFEL, ACKLEY, RASTRINGIN, GRIEWANK, SHUBER

\section{Experiment}

\subsection{Experiment 1}

\begin{itemize}
\item \textbf{Function:} Griewank
\item \textbf{Bats:} 256
\item \textbf{Iterations:} 10000
\end{itemize}


\subsection{Experiment 2}

\begin{itemize}
\item \textbf{Function:} Griewank
\item \textbf{Bats:} 768
\item \textbf{Iterations:} 10000
\end{itemize}

\section{Results}

\subsection{Device details}

The experiments were executed on a machine with the following configuration:

\textit{Intel(R) Core(TM) i5-4460  CPU @ 3.20GHz}
\textit{GK208 GeForce GT 720 1024 MB of vram}

\subsection{Experiment 1}

\begin{itemize}
    \item \textbf{CPU:} 1m4.888s
    \item \textbf{GPU:} 0m55.439s
\end{itemize}

\textbf{Speedup:} 1.16x

\subsection{Experiment 2}

\begin{itemize}
    \item \textbf{CPU:} 2m27.902s
    \item \textbf{GPU:} 0m21.976s
\end{itemize}

\textbf{Speedup:} 7x

\section{Conclusion}

It was observed speedups with big populations. The original BAT was proposed with 40 individuals and the speedups was seen with 250 individuals.
The advantages of the algorithm may be tested against a CPU implementation to be fair.
With this work it's clear that is possible to speedup the bat methaueristic using GPU but the best results are only achieved by a great population size.

\section{Further works}

It may be explored the usage of blocks as representation for the dimensions in which each bat deals.

\section{References}

\bibliographystyle{sbc}
\bibliography{paper}

\end{document}
