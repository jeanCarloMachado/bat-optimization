\documentclass[conference]{IEEEtran}

\usepackage[fleqn]{amsmath}
%\setlength{\mathindent}{0pt}
\interdisplaylinepenalty=2500



% *** SPECIALIZED LIST PACKAGES ***
%
%\usepackage{algorithmic}
% algorithmic.sty was written by Peter Williams and Rogerio Brito.
% This package provides an algorithmic environment fo describing algorithms.
% You can use the algorithmic environment in-text or within a figure
% environment to provide for a floating algorithm. Do NOT use the algorithm
% floating environment provided by algorithm.sty (by the same authors) or
% algorithm2e.sty (by Christophe Fiorio) as the IEEE does not use dedicated
% algorithm float types and packages that provide these will not provide
% correct IEEE style captions. The latest version and documentation of
% algorithmic.sty can be obtained at:
% http://www.ctan.org/pkg/algorithms
% Also of interest may be the (relatively newer and more customizable)
% algorithmicx.sty package by Szasz Janos:
% http://www.ctan.org/pkg/algorithmicx




% *** ALIGNMENT PACKAGES ***
%
%\usepackage{array}
% Frank Mittelbach's and David Carlisle's array.sty patches and improves
% the standard LaTeX2e array and tabular environments to provide better
% appearance and additional user controls. As the default LaTeX2e table
% generation code is lacking to the point of almost being broken with
% respect to the quality of the end results, all users are strongly
% advised to use an enhanced (at the very least that provided by array.sty)
% set of table tools. array.sty is already installed on most systems. The
% latest version and documentation can be obtained at:
% http://www.ctan.org/pkg/array


% *** SUBFIGURE PACKAGES ***
%\ifCLASSOPTIONcompsoc
%  \usepackage[caption=false,font=normalsize,labelfont=sf,textfont=sf]{subfig}
%\else
%  \usepackage[caption=false,font=footnotesize]{subfig}
%\fi
% subfig.sty, written by Steven Douglas Cochran, is the modern replacement
% for subfigure.sty, the latter of which is no longer maintained and is
% incompatible with some LaTeX packages including fixltx2e. However,
% subfig.sty requires and automatically loads Axel Sommerfeldt's caption.sty
% which will override IEEEtran.cls' handling of captions and this will result
% in non-IEEE style figure/table captions. To prevent this problem, be sure
% and invoke subfig.sty's "caption=false" package option (available since
% subfig.sty version 1.3, 2005/06/28) as this is will preserve IEEEtran.cls
% handling of captions.
% Note that the Computer Society format requires a larger sans serif font
% than the serif footnote size font used in traditional IEEE formatting
% and thus the need to invoke different subfig.sty package options depending
% on whether compsoc mode has been enabled.
%
% The latest version and documentation of subfig.sty can be obtained at:
% http://www.ctan.org/pkg/subfig


% *** FLOAT PACKAGES ***
%
%\usepackage{fixltx2e}
% fixltx2e, the successor to the earlier fix2col.sty, was written by
% Frank Mittelbach and David Carlisle. This package corrects a few problems
% in the LaTeX2e kernel, the most notable of which is that in current
% LaTeX2e releases, the ordering of single and double column floats is not
% guaranteed to be preserved. Thus, an unpatched LaTeX2e can allow a
% single column figure to be placed prior to an earlier double column
% figure.
% Be aware that LaTeX2e kernels dated 2015 and later have fixltx2e.sty's
% corrections already built into the system in which case a warning will
% be issued if an attempt is made to load fixltx2e.sty as it is no longer
% needed.
% The latest version and documentation can be found at:
% http://www.ctan.org/pkg/fixltx2e


%\usepackage{stfloats}
% stfloats.sty was written by Sigitas Tolusis. This package gives LaTeX2e
% the ability to do double column floats at the bottom of the page as well
% as the top. (e.g., "\begin{figure*}[!b]" is not normally possible in
% LaTeX2e). It also provides a command:
%\fnbelowfloat
% to enable the placement of footnotes below bottom floats (the standard
% LaTeX2e kernel puts them above bottom floats). This is an invasive package
% which rewrites many portions of the LaTeX2e float routines. It may not work
% with other packages that modify the LaTeX2e float routines. The latest
% version and documentation can be obtained at:
% http://www.ctan.org/pkg/stfloats
% Do not use the stfloats baselinefloat ability as the IEEE does not allow
% \baselineskip to stretch. Authors submitting work to the IEEE should note
% that the IEEE rarely uses double column equations and that authors should try
% to avoid such use. Do not be tempted to use the cuted.sty or midfloat.sty
% packages (also by Sigitas Tolusis) as the IEEE does not format its papers in
% such ways.
% Do not attempt to use stfloats with fixltx2e as they are incompatible.
% Instead, use Morten Hogholm'a dblfloatfix which combines the features
% of both fixltx2e and stfloats:
%
% \usepackage{dblfloatfix}
% The latest version can be found at:
% http://www.ctan.org/pkg/dblfloatfix


% *** PDF, URL AND HYPERLINK PACKAGES ***
%
%\usepackage{url}
% url.sty was written by Donald Arseneau. It provides better support for
% handling and breaking URLs. url.sty is already installed on most LaTeX
% systems. The latest version and documentation can be obtained at:
% http://www.ctan.org/pkg/url
% Basically, \url{my_url_here}.


% *** Do not adjust lengths that control margins, column widths, etc. ***
% *** Do not use packages that alter fonts (such as pslatex).         ***
% There should be no need to do such things with IEEEtran.cls V1.6 and later.
% (Unless specifically asked to do so by the journal or conference you plan
% to submit to, of course. )


% correct bad hyphenation here
\hyphenation{op-tical net-works semi-conduc-tor}

\begin{document}
\title{Bat Optimization on GPU}

\author{\IEEEauthorblockN{Jean Carlo Machado}
\IEEEauthorblockA{Univerisdade do Estado de Santa Catarina\\Computer Engineering\\
UDESC\\
Santa Catarina, Joinville,\\
Email: contato@jeancarlomachado.com.br}}

\maketitle
\begin{abstract}
    This works presents the Bat in GPU.
    The results show that...
\end{abstract}
\IEEEpeerreviewmaketitle

\section{Introduction}


The bat algorithm was introduced by \cite{original}. It uses the inspiration of microbas to look at their preys.

Many populational optimization algorithms can benefit from paralization.

This work attemps to investigate the applicability of GPU parallization
on the bat algorithm. Previously some demonstrations of the bat
algorithm paralelized on CPU were presented (reference referece),
however, til the day of this publication no implementation of the bat
algorithm was found on GPU.

It was developed two versions of the algorithm. One that runs on CPU and
the other which uses CPU.

\hfill mds
\hfill December 21, 2016

\section{Bat Design on CPU}

In this work the bath algorithm used was the one proposed by
\cite{parpinelli}, since it represents a concrete demonstration of how
the bat metha-heuristic.

The CPU version was single threaded.

The random algorightm used was the mersenne twister.

\subsection{Pseudo-code}

\begin{flalign}
\begin{split}
\label{Pseudo Code}
\text{Parameters:} n, \alpha\, \lambda\\
\text{Initialize Bats}\\
\text{Evaluate fitness bats}\\
\text{Selects best}\\
\text{\textbf{while} stop criteria false \textbf{do}}\\
    \text{\textbf{for} i=1\ to\ n\ \bf do}\\
        f_i=f_{min} + (f_{max} - f_{min})\beta, \in \beta [0,1]\\
        \vec{v}_i^{t+1} = \vec{v}_i^{t} + (\vec{x}_i^{t} + \vec{x}_*^{t})f_i\\
        \vec{x}_{temp} = \vec{x}_i^{t} + \vec{v}_i^{t+1}\\
        \text{\bf if } rand < r_i, rand \in [0,1] \text{\textbf{then} {local search}}\\
            \vec{x}_{temp} = \vec{x}_* + \epsilon A_m, \epsilon \in [-1, 1]\\
        \text{\bf end if}\\
        \text{Single dimension perturbation in} x_{temp}\\
        \text{\bf if} a < A_i^t , a \in [0,1]\\
        \vec{x}_i^t = \vec{x}_{temp}\\
        r_i = exp(\lambda * i)\\
        A_i =  A_{0} * \alpha^i\\
        \text{\bf end if}\\
        \text{Selects best}\\
    \text{\bf end for}\\
\text{\bf end while}
\end{split}
\end{flalign}

\section{Bat Design on GPU}

Since the BAT agorithm uses a population of bats, the most intuitive
parallization method to apply on it is to use each bat on a GPU core.
\cite{pso-gpu} used a similar method for a GPU implementation for the PSO algorithm.
For the GPU version the approach used was the split of each individual in one thread.

\subsection{Pseudo-code GPU}
\begin{flalign}
\begin{split}
\label{Pseudo Code}
\text{Parameters:} n, \alpha\, \lambda\\
\text{Initialize bats assicrously}\\
\text{syncronize threads}\\
\text{Evaluate fitness bats}\\
\text{Selects best}\\
\text{\textbf{while} stop criteria false \textbf{do}}\\
    \text{for each thread } i \\
        f_i=f_{min} + (f_{max} - f_{min})\beta, \in \beta [0,1]\\
        \vec{v}_i^{t+1} = \vec{v}_i^{t} + (\vec{x}_i^{t} + \vec{x}_*^{t})f_i\\
        \vec{x}_{temp} = \vec{x}_i^{t} + \vec{v}_i^{t+1}\\
        \text{\bf if } rand < r_i, rand \in [0,1] \text{\textbf{then} {local search}}\\
            \vec{x}_{temp} = \vec{x}_* + \epsilon A_m, \epsilon \in [-1, 1]\\
        \text{\bf end if}\\
        \text{Single dimension perturbation in} x_{temp}\\
        \text{\bf if} a < A_i^t , a \in [0,1]\\
        \vec{x}_i^t = \vec{x}_{temp}\\
        r_i = exp(\lambda * i)\\
        A_i =  A_{0} * \alpha^i\\
        \text{\bf end if}\\
        \text{syncronize threads}\\
        \text{Selects best}\\
    \text{\bf end for}\\
\text{\bf end while}
\end{split}
\end{flalign}

\section{Experiments}

The benchmark functions used were the following:

\begin{itemize}
    \item Ackley
    \item Griewank
    \item Rastringin
    \item Rosenbrook
\end{itemize}


The experiments were executed on a machine with the following configuration:

\textit{Intel(R) Core(TM) i5-4460  CPU @ 3.20GHz}
\textit{GK208 GeForce GT 720 1024 MB of vram}

Each experiment runned a total of 1x10^4\ times.

\begin{table}[!t]
    \renewcommand{\arraystretch}{1.3}
    \caption{Experiments}
    \label{experiments}
    \centering
    \begin{tabular}{c|c|c|c}
    \hline
        \bf Name & Function &  Dimensions & Agents\\
    \hline
        E1 & Ackley & 100 & 256\\
        E2 & Ackley & 100 & 768\\
        E3 & Griewank & 100 & 256\\
        E4 & Griewank & 100 & 768\\
        E5 & Rastringin & 100 & 256\\
        E6 & Rastringin & 100 & 768\\
        E7 & Rosenbrook & 100 & 256\\
        E8 & Rosenbrook & 100 & 768\\
    \end{tabular}
\end{table}

\section{Results}


\begin{table}[!t]
    \renewcommand{\arraystretch}{1.3}
    \caption{Results}
    \label{results}
    \centering
    \begin{tabular}{c|c|c|c}
    \hline
        \bf Name & Total CPU & Total GPU & Speedup\\
    \hline
        E3 & 1m4.888s & 0m55.439s & 1.16x\\
        E4 & 2m27.902s & 0m21.976s & 7x
    \end{tabular}
\end{table}


\section{Conclusion}

It was observed speedups with big populations. The original BAT was proposed with 40 individuals and the speedups was seen with 250 individuals.
The advantages of the algorithm may be tested against a CPU implementation to be fair.
With this work it's clear that is possible to speedup the bat methaueristic using GPU but the best results are only achieved by a great population size.

\section{Further works}

It may be explored the usage of blocks as representation for the dimensions in which each bat deals.


\begin{thebibliography}{1}

\bibitem{original}
    Xin-She Yang \emph{A New Metaheuristics Bat-Inspired Algorithm}. Department of Engineering, Cambridge, 2010.

\bibitem{parpinelli}
    Jelson A. Cordeiro, Rafael Stubs Parpinelli Heitor Silvério Lopes \emph{Análise de Sensibilidade dos Parâmetros do Bat Algorithm e Comparação de Desempenho}.
\bibitem{pso-gpu}
    PSO-GPU: Accelerating Particle Swarm Optimization in CUDA-Based Graphics Processing Unit

\end{thebibliography}

\end{document}
